
\documentclass[11pt,a4paper]{article}
\usepackage{jheppub,amsmath, amsthm, amssymb,slashed,url,diagram,bm}
\usepackage{graphicx}
\usepackage{epstopdf}
\def\t{\widetilde} 
\def\ct{{\cmmib t}}
\def\btau{{\bm \tau}}
\def\s{{s}}
\def\tr{\mathrm{Tr}}
\def\NSthree{{\mathrm{NS}}^3}
\def\NSRR{{\mathrm{NS}}\cdot{\mathrm{R}}^2}
\def\be{\begin{equation}}
\def\ee{\end{equation}}
\def\bt{{{t}}}
\def\frakn{{\mathfrak N}}
\def\frakl{{\mathfrak L}}
\def\eusmn{{\eusm N}}
\def\Im{{\mathrm{Im}}}
\def\PGL{{\mathrm{PGL}}}
\def\OSp{{\mathrm{OSp}}}
\def\hat{\widehat}
\def\tilde{\widetilde}
\def\frak{\mathfrak}
\def\D{{\mathcal D}}
\def\S{{\mathcal S}}
\def\RP{{\Bbb{RP}}}
\def\NS{{\mathrm{NS}}}
\def\Ra{{\mathrm{R}}}
\def\W{{\mathcal W}}
\def\WW{\mathfrak W}
\def\MM{\mathfrak M}
\def\euq{{\eusm F}}
\def\O{{\mathcal O}}
\def\V{{\mathcal V}}
\def\k{{\cmmib k}}
\def\w{{u}}
\def\dzzt{\D(\t z,z|\theta)}
\def\dzztt{\D(\t z,z|\t\theta,\theta)}
\def\Bbb{\mathbb}
\def\SIgma{\Sigma}
\def\red{{\mathrm{red}}}
\def\A{{\mathcal A}}
\def\d{{\mathrm d}}
\def\tW{{W}}
\def\intt{{\mathrm{int}}}
\def\b{\overlineAn}
\def\R{{\mathbb R}}
\def\RR{{\mathcal R}}
\def\C{{\mathbb C}}
\def\U{{\mathcal U}}
\def\D{{\mathcal D}}
\def\B{{\mathcal B}}
\def\G{{\mathcal G}}
\def\[{\bigl [}
\def\Spin{{\mathrm{Spin}}}
\def\]{\bigr ]}
\def\CP{{\mathbb{CP}}}
\def\N{{\mathcal N}}
\def\T{{\mathcal T}}
\def\p{{'}}
\def\Z{{\mathbb Z}}
\def\Q{{\mathbb Q}}
\def\half{{\frac{1}{2}}}
\def\CC{{\mathcal C}}
\def\ad{{\mathrm{ad}}}
\def\Btriv{{\mathcal B}_{\mathrm{triv}}}
\def\L{{\eusm L}}
\def\h{\hat }
\def\h{\widehat}
\def\Bcc{{\mathcal B_{\mathrm{cc}}}}
\def\K{{\mathcal K}}
\def\V{{\mathcal V}}
\def\J{{\mathcal J}}
\def\I{{\mathcal I}}
\def\P{{\mathcal P}}
\def\B{{\mathcal B}}
\def\Boper{{\mathcal B_{\mathrm{oper}}}}
\def\M{{\mathcal M}}
\def\W{{\mathcal W}}
\def\X{\mathcal X}
\def\Y{\mathcal Y}
\def\P{{\mathcal P}}
\def\l{\langle}
\def\r{\rangle}
\def\H{{\mathcal H}}
\def\ZZ{\eusm B}
\def\cZ{\mathcal Z}
\def\sW{{ W}}
\def\epsilon{\varepsilon}
\def\sY{{ Y}}
\def\nY{{\mathcal Y}}
\def\ca{{\cmmib a}}
\def\ss{{d}}
\def\sf{{\mathfrak s}}
\def\e{{\mathbf e}}
\def\i{{\mathbf i}}
\def\spin{{\mathrm{spin}}}
\def\m{\cmmib m}
\def\tilde{\widetilde}
\def\bar{\overline}
\def\neg{\negthinspace}
\def\Ber{{\mathrm {Ber}}}
\def\BBer{{\text{\it{Ber}}}}
\def\TT{{\mathrm T}}
\def\Tr{{\mathrm {Tr}}}
\def\E{\mathbb{E}}

\font\teneurm=eurm10 \font\seveneurm=eurm7 \font\eighteurm=eurm8 \font\fiveeurm=eurm5
\newfam\eurmfam
\textfont\eurmfam=\teneurm \scriptfont\eurmfam=\seveneurm
\scriptscriptfont\eurmfam=\fiveeurm
\def\eurm#1{{\fam\eurmfam\relax#1}}
%\font\teneurm=eurm10 \font\seveneurm=eurm7 \font\fiveeurm=eurm5
%\newfam\eurmfam
%\textfont\eurmfam=\teneurm \scriptfont\eurmfam=\seveneurm
%\scriptscriptfont\eurmfam=\fiveeurm
%\def\eurm#1{{\fam\eurmfam\relax#1}}
\font\teneusm=eusm10 \font\seveneusm=eusm7 \font\fiveeusm=eusm5
\newfam\eusmfam
\textfont\eusmfam=\teneusm \scriptfont\eusmfam=\seveneusm
\scriptscriptfont\eusmfam=\fiveeusm
\def\eusm#1{{\fam\eusmfam\relax#1}}
\font\tencmmib=cmmib10 \skewchar\tencmmib='177
\font\sevencmmib=cmmib7 \skewchar\sevencmmib='177
\font\fivecmmib=cmmib5 \skewchar\fivecmmib='177
\newfam\cmmibfam
\textfont\cmmibfam=\tencmmib \scriptfont\cmmibfam=\sevencmmib
\scriptscriptfont\cmmibfam=\fivecmmib
\def\cmmib#1{{\fam\cmmibfam\relax#1}}
\def\F{\eusm F}
\def\Pi{\varPi}
\def\g{\text{{\teneurm g}}}
\def\sg{\text{{\eighteurm g}}}
\def\ssg{\text{{\seveneurm g}}}
\def\n{\text{{\teneurm n}}}
\def\sn{\text{{\eighteurm n}}}
\def\ssn{\text{{\seveneurm n}}}
\def\m{\text{{\teneurm m}}}
\def\sm{\text{{\eighteurm m}}}
\def\ssm{\text{{\seveneurm m}}}
\newtheorem{theorem}{Theorem}[subsection]
\newtheorem{lemma}{Lemma}[subsection]
\newtheorem{definition}{Definition}[subsection]
\newtheorem{problem}{Problem}

\title{Paper Summary}

 \author{Yu Chen}
\affiliation{The Chinese University of HongKong, Department of Mechanical and Automation Engineering}
\emailAdd{anschen@link.cuhk.edu.hk}




\begin{document}\maketitle
%-------------------------------------------------------------------------------------------------------------------------------------------------
\section{Adapting to a Market Shock: Optimal Sequential Market-Making}
The main result of this article is optimal bid-ask price under the strong assumption given by author.
\subsection{Model}
We use random variable $V$, obeying $p_V(v)$, to denote the underlying value of asset. At time $t$, the markert-maker raise its bid price $b_t$ and ask price $a_t$. Then a trader comes in to find whether there exists some trading chance. Besides, the trader expect the value of asster is $w_t = V + \epsilon$, where $\epsilon$ is random variable with CDF $F_{\epsilon}$. For example, if $w_t < b_t$, then trader will sell asset; if $w_t > a_t$, then trader will buy the asset. Furthermore, we use $x_t$ to denote the signal such that $x \in \{-1,0,1\}$ means trader buying, no action, selling. Hence, we can find signal distribution as,
\begin{equation}
p(x|v;a_t,b_t) = \left\{\begin{array}{cc} F_{\epsilon}(b_t-v) & x = 1\\   F_{\epsilon}(a_t - v)- F_{\epsilon}(b_t -v) &x= 0\\  1 - F_{\epsilon}(a_t -v) &x = -1\\ 
\end{array}\right.
\end{equation}
If the bid price is aceepted at time $t$, we can obtain a bid reward,
\begin{equation}
r_t^{\mathrm{bid}} = \int dv(v-b_t) F_{\epsilon}(b_t -v)p_t(v) = \int dv v F_{\epsilon}(-v)p_t(v+b_t)
\end{equation}
Similarly, we can obtain an ask reward,
\begin{equation}
r_t^{\mathrm{ask}} = \int dv(a_t -v)(1-F_{\epsilon}(a_t-v))p_t(v) = \int dv v F_{\epsilon}(-v)p_t(a_t-v),
\end{equation}
where we have used the assumption that distribution of $\epsilon$ is symmetric, i.e. $F_{\epsilon}(x) = 1 - F_{\epsilon}(-x)$.
Hence,
\begin{align}
r_t & = r_t^{\mathrm{bid}} + r_t^{\mathrm{ask}} \\ 
& = \int dv v F_{\epsilon}(-v)(p_t(v+b_t)+p_t(a_t-v)).
\end{align}
For a zero-profit market, the sequential decision $\{a_t,b_t\}$ is very simple, given by equations $r_t^{\mathrm{bid}} = r_t^{\mathrm{ask}} = 0$, which means,
\begin{align}
& b_t = \frac{\int dv v F_{\epsilon}(b_t-v)p_t(v)}{\int dv F_{\epsilon}(b_t -v)p_t(v)}, \\ 
& a_t = \frac{\int dv v F_{\epsilon}(v-a_t)p_t(v)}{\int dv F_{\epsilon}(v-a_t)p_t(v)}.
\end{align}
However, more interesting thing is that we want to be a monopolist in this market, which means we should maximize our reward.
\begin{equation}
\frac{\partial r_t^{\mathrm{ask}}}{\partial a_t} = 0, \frac{\partial r_t^{\mathrm{bid}}}{\partial b_t} = 0.
\end{equation}
Analytically, it is,
\begin{align}
& b_t = \frac{\int dv p_t(v)(vf_{\epsilon}(b_t-v)-F_{\epsilon}(b_t-v))}{\int dv p_t(v)f_{\epsilon}(b_t-v)}, \\ 
& a_t = \frac{\int dv p_t(v)(vf_{\epsilon}(a_t-v)+F_{\epsilon}(v-a_t))}{\int dv p_t(v)f_{\epsilon}(a_t-v)},
\end{align}
where $f_{\epsilon}(x) = F_{\epsilon}'(x)$. Here is a problem stated in as: If you believer your current knowledge of random variable $V$, i.e. $p_t(v)$, you can choose the short time optimal decision at least numerically. However, the belief of $p_t(v)$ should not be very high, we also need to choose other decisions such that we can update our knowledge of $V$ more quickly. This is `explore-exploit' problem. Formally, we can restate our problem in more mathematical language instead of maximizing our local reward. Our aim to become a monopolist is to maximize the following value function,
\begin{equation}
V(p_t;\pi) = \E[r_t|p_t,a_t^{\pi}(p_t),b_t^{\pi}(p_t)] + \gamma V(p_{t+1};\pi). 
\end{equation}



























\end{document}